\documentclass[a4paper,14pt]{article}
\usepackage[utf8]{inputenc}


\usepackage{graphicx}
\usepackage{amsmath}
\usepackage{amsfonts}
\usepackage{amssymb}
\usepackage{subcaption}
\usepackage{hyperref}
\usepackage{xcolor}
\usepackage{tabularx}
    \newcolumntype{L}{>{\raggedright\arraybackslash}X}



\definecolor{light-gray}{gray}{0.92}

\usepackage{listings}

\lstset{
basicstyle=\scriptsize\ttfamily,
columns=flexible,
breaklines=true,
backgroundcolor = \color{light-gray}
}


\usepackage[margin=1.3in]{geometry}

\usepackage{etoolbox}
\makeatletter
\patchcmd{\@verbatim}
  {\verbatim@font}
  {\verbatim@font\small}
  {}{}
\makeatother

\let\endtitlepage\relax

% Title Page
\title{Commands for hoseclient.py}
\author{J. Barrett}


\begin{document}
\maketitle


\section{Operation at the 37m}

The command client software hoseclient.py allows the user to issue commands to the gpu-spectrometer via ZeroMQ TCP messages. Currently this software is hard-coded to be run directly on the gpu-spectrometer machine,
as it communicates with the command server via the local-host IP:PORT of 127.0.0.1:12345. However, if the hoseclient.py code is installed on another machine the IP:PORT could be configured to send messages remotely to the server
without first logging into the gpu-spectrometer by modifying the file hclient\_module.py.

In the client interface provides the following commands for interactive sessions.
\begin{enumerate}
 \item Start a recording immediately for an unknown duration without specifying an experiment, source or scan name (the default values for the experiment and scan name will be: ExpX, and DDD-HHMMSS): \verb|record=on|.
 \item Start a recording immediately for a specified duration (in seconds) without specifying an experiment, source or scan name (the default values: ExpX, SrcX, and DDD-HHMMSS will be used): e.g. for 30 seconds \verb|record=on:30|. Note: The command prompt
 will only return after the recording is finished.
 \item Start recording for an unknown duration and specify an experiment, source and scan name: \verb|record=on:experiment_name:source_name:scan_name|.
 \item Start recording at a specific start time and duration and specify an experiment, source and scan name: \verb|record=on:experiment_name:source_name:scan_name:YYYYDDDHHMMSS:duration|. Duration must be given in seconds, and the time in UTC.
 \item Stop a recording immediately: \verb|record=off|.
 \item Get the current recording state (on/of): \verb|record?|.
 \item Set the upper and lower bin numbers to be used when summing power in the spectrally binned power measurement: e.g. \verb|set_power_bins=65000:65100|. Valid bin numbers are between 0 and 1048576 in the default ADQ7 configuration. In order to determine
 the appropriate bin numbers for a particular spectral feature it may be helpful to inspect some data from a test scan with the \verb|SpectrumPlot| with the \verb|-b| option enabled to plot the spectrum as a function of bin number.
  \item DEPRECATED: Start the export of spectrometer log data to the InfluxDB database (applicable only to Westford setup): \verb|startlog2db|.
 \item List available commands: \verb|help|.
 \item Quit the client interface, but leave the spectrometer daemon running: \verb|quit|.
 \item Quit the client interface and shutdown the spectrometer daemon entirely: \verb|shutdown|.
 
\end{enumerate}

All spectrometer data files are stored under the data directory: \verb|/media/oper/data/|, and unless specified scans will default to being saved under the default experiment directory \verb|/media/oper/data/ExpX|.


\end{document}          
